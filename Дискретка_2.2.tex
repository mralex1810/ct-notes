\documentclass{article}
\input{Headers/header}
\input{Headers/formal}
\input{Headers/graphs}

\fancyhead[L]{Дискретная математика}

\begin{document}
    \tableofcontents
    \section{Производящие функции.}
    \begin{remark}
        Вообще производящая функция~--- не очень такое название. Потому что это скорее не функция, а способ записи бесконечных последовательностей.\\
        Вот есть у нас $2,4,8,16,\ldots$ и сразу понятно, что имеется ввиду. А вот когда мы видим $1,2,5,14,42$ и знающие люди поймут, что это, скорее всего, числа Каталана. Но всё равно это не то чтобы однозначно определяет, что мы имеем ввиду.\\
        Можно записать $a_n=2^n$ и сразу станет понятно, что это степени двойки. А когда мы запишем $a_n=C_n$, то поймут точно не все.\\
        Короче, работать с такими вещами не очень приятно. И люди задумались: а как компьютеру дать бесконечные последовательности, чтобы он их понял.\\
        И люди нашли инструмент из теории вероятности и статистики~--- собственно, производящие функции.
    \end{remark}
    \begin{definition}
        Пусть $\{a_n\}_{n=0}^\infty$~--- последовательность. Тогда \textbf{формальный степенной ряд} этой последовательности~--- это запись вида
        $$
        A(t)=a_0t^0+a_1t^1+a_2t^2+\cdots+a_nt^n+\cdots
        $$
    \end{definition}
    \begin{remark}
        Вот есть у нас многочлен: $t^2+t-7$. Что такое $t$? Ну, по сути, буква. Она не обладает значением, мы можем только потом уже рассмотреть многочлен как многочлен над каким-то кольцом, и уже значения его анализировать.\\
        Вот и тут по сути мы имеем просто букву $t$, она ничему не равна.
    \end{remark}
    \begin{claim}
        Очевидно, формальные степенные ряды биективно сопоставляются последовательностям.
    \end{claim}
    \begin{remark}
        Пока что формальный степенной ряд ничем не лучше просто последовательности. Но на самом деле с формальными степенными рядами можно производить полезные операции, которые позволят нам конечным количеством символов описать интересующие нас последовательности.
    \end{remark}
    \begin{example}
        Какие формальные степенные ряды мы уже можем легко записать? Ну, те, которые соответствуют многочленам. То есть формальные степенные ряды тех последовательностей, которые имеют конечное количество ненулевых элементов.
    \end{example}
    \begin{definition}
        Суммой формальных степенных рядов называется степенной ряд суммы их последовательностей.\\
        Умножением формального ряда на число называется произведение его последовательности и этого числа.
    \end{definition}
    \begin{remark}
        Понятно, что это согласуется с тем, как мы могли бы сложить ряды.\\
        Но ведь ряды ещё можно умножать. Что получим?
    \end{remark}
    \begin{definition}
        \textbf{Произведением формальных степенных рядов}
        $$
        A(t)=a_0t^0+a_1t^1+a_2t^2+\cdots+a_nt^n+\cdots\qquad B(t)=b_0t^0+b_1t^1+b_2t^2+\cdots+b_nt^n+\cdots
        $$
        называется ряд
        $$
        (AB)(t)=C(t)=c_0+c_1t^1+c_2t^2+\cdots+c_nt^n+\cdots\qquad c_k=\sum\limits_{j=0}^ka_jb_{k-j}
        $$
    \end{definition}
    \begin{remark}
        А делить как? Ну, если $\frac{A(t)}{B(t)}=C(t)$, то $A(t)=(BC)(t)$. Ну,
        $$
        a_0=b_0c_0\Rightarrow c_0=\frac{a_0}{b_0}
        $$
        $$
        a_1=b_1c_0+b_0c_1\Rightarrow c_1=\frac{a_1-c_0b_1}{b_0}
        $$
    \end{remark}
    \begin{definition}
        \textbf{Частным формальных степенных рядов}
        $$
        A(t)=a_0t^0+a_1t^1+a_2t^2+\cdots+a_nt^n+\cdots\qquad B(t)=b_0t^0+b_1t^1+b_2t^2+\cdots+b_nt^n+\cdots
        $$
        где $b_0\neq0$ называется ряд
        $$
        (\frac AB)(t)=C(t)=c_0+c_1t^1+c_2t^2+\cdots+c_nt^n+\cdots\qquad c_k=\frac{a_k-\sum\limits_{j=0}^{k-1}c_jb_{k-j}}{b_0}
        $$
    \end{definition}
    \begin{example}
        $$
        \frac1{1-t-t^2}
        $$
        Так, $c_0=1$,
        $$
        c_1=\frac{a_1-c_0b_1}{b_0}=1
        $$
        $$
        c_2=\frac{a_2-c_1b_1-c_0b_2}{b_0}=2
        $$
        Давайте в общем виде, учитывая тот факт, что $b_k=0$ для $k>2$,
        $$
        c_n=a_n-c_{n-1}b_1-c_{n-2}b_2=c_{n-1}+c_{n-2}
        $$
        Да это же числа Фибоначчи!
    \end{example}
    \begin{claim}
        Если $a_n\in\mathbb Z$, $b_n\in\mathbb Z$, $b_0=\pm1$, $C(t)=\frac{A(t)}{B(t)}$, то $c_n\in\mathbb Z$.
    \end{claim}
    \begin{proof}
        Очевидно.
    \end{proof}
    \begin{example}
        Давайте получим $2^n$! Что нам хочется?
        $$
        P(t)=1+2t+4t^2+\cdots+2^nt^n+\cdots
        $$
        Хм-м-м-м. Может, так:
        $$
        P(t)=(2t)^0+(2t)^1+(2t)^2+\cdots+(2t)^n+\cdots
        $$
        Хм-м-м, кажется, геометрическая прогрессия.
        $$
        P(t)=\frac1{1-2t}
        $$
        Ок?
    \end{example}
    \begin{remark}
        Ну, очень хочется так думать, но вообще так жить некорректно, неверно интерпретировать $t$ как число. Потому что если мы будем, то мы придём в мат. анализ и вспомним о том, что у рядов есть радиус сходимости, и если ряд условно сходится или расходится, то мы проиграли.\\
        Но почему производящие функции использовались в мат. статистике? Потому что к ним часто применяется следующий приём: давайте сделаем то, что делать нельзя, получим что-то, а потом как-нибудь иным способом докажем, что наш ответ норм.
    \end{remark}
    \begin{example}
        Мы получили
        $$
        P(t)=\frac1{1-2t}
        $$
        Давайте проверим, что подходит.
        $$
        a_0=1,a_{k\geqslant1}=0\qquad b_0=1,b_1=-2,b_{n\geqslant2}=0
        $$
        $$
        c_0=\frac{a_0}{b_0}=1\qquad c_1=\frac{a_1-c_0b_1}{b_0}=2\qquad c_n=-c_{n-1}b_1=2c_{n-1}
        $$
        Действительно, подходит.
    \end{example}
    \begin{remark}
        Почему мы так делаем вообще? Потому что у нас для некоторых $t$ верна формула
        $$
        (2t)^0+(2t)^1+(2t)^2+\cdots+(2t)^n+\cdots=\frac1{1-2t}
        $$
        И если бы мы получили иной ряд на самом деле при делении, мы бы получили, что указанная выше формула не верна нигде. Есть патологические примеры (см. математический анализ, функция со всеми нулевыми производными, не равная тождественно нулю), но в целом обычно получается жить в ситуации, когда мы нарушаем правила математики, а потом доказываем, что получили верный ответ.
    \end{remark}
    \begin{example}
        Хорошо, давайте построим числа Каталана. Мы знаем, что
        $$
        C_n=\sum\limits_{i=0}^{n-1}C_iC_{n-i-1}
        $$
        Та-а-а-ак, что-то знакомое. Пусть $C(t)$~--- числа Каталана. Тогда из формулы выше хотелось бы, чтобы получилось
        $$A(t)=C(t)C(t)$$
        Где $A(t)$~--- числа Каталана со сдвинутыми коэффициентами. Как нам сдвинуть коэффициенты? Умножить на $t$:
        $$
        C(t)=C(t)C(t)t
        $$
        Это почти хорошо, разве что тут нулевой коэффициент получится ноль, а нам надо 1:
        $$
        C(t)=C(t)C(t)t+1
        $$
        Так, ну, хорошо, начинаем делать грязь:
        $$
        C(t)=\frac{1\pm\sqrt{1-4t}}{t}
        $$
        Тут всё плохо. Тут и $\pm$, и корень формального ряда и деление на $t$, а на $t$ делить нельзя т.к. у него нулевой коэффициент ноль.\\
        Ну, делаем грязь дальше. Что делать с корнем? По Тейлору раскладывать, конечно же
        $$
        (1+x)^\alpha=1+\frac\alpha1x+\frac{\alpha(\alpha-1)}2x^2+\cdots+\frac{\alpha(\alpha-1)\cdot\cdots\cdot(\alpha-n+1)}{n!}x^n+\cdots
        $$
        Это мы так легко не докажем, но у нас $\alpha$ конкретное ($\frac12$), и вот верность этой формулы для него доказать можно довольно легко.
        $$
        \sqrt{1-4t}=1-\frac{1/2}14t+\frac{(1/2)(-1/2)}216t^2-\frac{(1/2)(-1/2)(-3/2)}664t^3+\frac{(1/2)(-1/2)(-3/2)(-5/2)}{24}256t^4+\cdots
        $$
        Хорошо, давайте попытаемся это посчитать. Получим
        $$
        1-2t-4t^2-10t^5-28t^5-\cdots
        $$
        Теперь оставшиеся две проблемы: $\pm$ и деление на $t$. Посмотрим на второе. Почему мы не хотели делить на $t$? Потому что у нас был $a_0$, который не поделить на $0$. Но если свободного члена нет, то будет логично делить на $t$. Так понятно, что нам надо взять $\pm$ как $-$. И мы, как нетрудно заметить, получим числа Каталана.
    \end{example}
    \begin{remark}
        Хорошо, какие ещё у нас есть операции? Ну, интегрирование и дифференцирование. Например, что будет, если мы хотим каждый коэффициент умножить на его номер? Ну, очевидно, так:
        $$
        B(t)=A'(t)\cdot t
        $$
    \end{remark}
    \begin{definition}
        \textbf{Производной формального степенного ряда} называется понятно, что.
    \end{definition}
    \begin{property}
        Несложно проверить формулы производной произведения и производной частного для формальных степенных рядов.
    \end{property}
    \begin{definition}
        Интегралом формального степенного ряда называется также понятно, что, разве что константа интегрирования равна нулю. Обозначение:
        $$
        \int A(t)
        $$
    \end{definition}
    \begin{remark}
        Из-за последнего (конкретной константы интегрирования) интегралами пользуются довольно нечасто.
    \end{remark}
    \begin{property}
        Несложно проверить, что верна формула интегрирования по частям.
    \end{property}
    \begin{remark}
        Ну что, подставляем один ряд в другой?\\
        Пусть есть
        $$A(t)=a_0+a_1t+a_2t^2+\cdots+a_nt^n+\cdots$$
        $$B(t)=b_0+b_1t+b_2t^2+\cdots+b_nt^n+\cdots$$
        Тогда что такое $A(B(t))$?
        $$
        C(t)=A(B(t))=a_0+a_1(b_0+b_1t+b_2t^2+\cdots+b_nt^n+\cdots)+a_2(b_0+b_1t+b_2t^2+\cdots+b_nt^n+\cdots)^2+\cdots
        $$
        Ну и что с этим делать? У нас даже свободный член нормально не считается, там получится $a_0+a_1b_0+a_2b_0^2+\cdots$. Это вообще какая-то сумма ряда, а это, во-первых, матан, во-вторых, радиусы сходимости и прочий ужас.\\
        Нам не нравится $b_0$, пусть подставлять можно только ряд с $b_0=0$. Тогда $c_0=a_0$. Чему равно $c_1$? Ну, во второй скобке там степени не ниже квадрата. Значит
        $$
        c_1=a_1b_1
        $$
        А $c_2$? Ну,
        $$
        c_2=a_1b_2+a_2b_1^2
        $$
        Пока непонятно, давайте запишем $c_3$:
        $$
        c_3=a_1b_3+a_2(b_1b_2+b_2b_1)+a_3b_1^3
        $$
        Кринж какой-то, но уже что-то более понятное.
    \end{remark}
    \begin{definition}
        Пусть
        $$A(t)=a_0+a_1t+a_2t^2+\cdots+a_nt^n+\cdots$$
        $$B(t)=b_1t+b_2t^2+\cdots+b_nt^n+\cdots$$
        Тогда \textbf{подстановкой формального степенного ряда} $B$ \textbf{в ряд} $A$ называется ряд
        $$
        C(t)=c_0+c_1t+c_2t^2+\cdots+c_nt^n+\cdots\qquad c_n=\sum\limits_{k=0}^na_k\sum\limits_{n=i_1+\cdots+i_k}\prod\limits_{j=1}^kb_{i_j}
        $$
    \end{definition}
    \begin{theorem}
        Пусть дана последовательность $a_n$. Следующие три условия эквивалентны:
        \begin{enumerate}
            \item $A=\frac PQ$, где $P$ и $Q$~--- многочлены.
            \item $a_n$ задаётся рекуррентным соотношением:
            $$
            \forall n\geqslant m~a_n=c_1a_{n-1}+c_2a_{n-2}+\cdots+c_ka_{n-k}
            $$
            Где $m\geqslant k$~--- некоторые натуральные числа, $c_j$~--- некоторые вещественные числа.
            \item Для некоторого $n_0\in\mathbb N$ выполнено
            $$
            \forall n>n_0~a_n=\sum\limits_{i=1}^sp_i(n)r_i^n
            $$
            Где $s$~--- некоторое натуральное число, $r_i\in\mathbb C$.
        \end{enumerate}
    \end{theorem}
    \begin{proof}
        \begin{itemize}
            \item[$1\to2$] Известно, что
            $$
            A(t)=\frac{p_0+p_1t+\cdots+p_mt^m}{q_0+q_1t+\cdots+q_kt^k}
            $$
            Известно, что $q_0\neq0$, иначе нельзя делить. Далее Н.У.О. $q_0=1$ (иначе поделим числитель и знаменатель на $q_0$).\\
            Давайте делить:
            \begin{align*}
                a_0&=p_0\\
                a_1&=p_1-q_1a_0\\
                \cdots&
            \end{align*}
            Дальше у нас что-то кончится раньше, либо $k$, либо $m$. Пусть, например, $k$:
            \begin{align*}
                a_k&=p_k-q_1a_{k-1}-q_2a_{k-2}-\cdots-q_{k}a_0\\
                a_{k+1}&=p_{k+1}-q_1a_{k}-q_2a_{k-1}-\cdots-q_{k}a_1-\underbrace{q_{k+1}a_0}_0\\
                \cdots&
            \end{align*}
            Потом у нас кончится $m$. И дальше будет то же самое, но без $p_n$. А это как раз условие 2.
            \item[$2\to1$] Известно
            $$
            A(t)=a_0+a_1t+a_2t^2+\cdots
            $$
            Домножим это на некоторые штуки:
            \begin{align*}
                A(t)=a_0+a_1t+&a_2t^2+\cdots\\
                A(t)t=a_0t+&a_1t^2+a_2t^3+\cdots\\
                A(t)t^2={}&a_0t^2+a_1t^3+a_2t^4+\cdots
            \end{align*}
            А теперь давайте возьмём нашу рекурренту и заменим $a_{n-j}$ на $A(t)t^j$:
            $$
            A(t)-c_1A(t)t-c_2A(t)t^2-\cdots-c_kA(t)t^k
            $$
            У нас тогда по рекуррентному соотношению все коэффициенты при $t^{>m}$ будут ноль. То есть разность получится равной какому-то многочлену $m$-той степени. Ну так извините, получим
            $$
            A(t)=\frac{P(t)}{1-c_1t-c_2t^2-\cdots-c_kt^k}
            $$
        \end{itemize}
    \end{proof}
    \begin{remark}
        Из этого мы можем решить такую задачу: пусть есть последовательность, заданная линейной рекуррентой. Нам надо узнать, нельзя ли уменьшить рекурренте порядок.\\
        Ну, как мы выяснили, порядок рекурренты равен степени знаменателя, а значит надо представить нашу последовательность как частное двух последовательностей и сократить максимально сильно. Чтобы сократить, надо научиться искать НОД двух многочленов (алгоритмом Евклида, например).
    \end{remark}
    \begin{remark}
        Другая задачка: возьмём рекурренту и найдём $a_n$.
        $$
        a_n=c_1a_{n-1}+c_2a_{n-2}+\cdots+c_ka_{n-k}
        $$
        Запишем вот такое (очевидно, выполненное) равенство:
        $$
        \matr{
            a_{n}\\
            a_{n-1}\\
            \vdots\\
            a_{n-k+2}\\
            a_{n-k+1}
        }=\matr{
            c_1 & c_2 & \cdots & c_{k-1} & c_k\\
            1 & 0 & \cdots & 0 & 0\\
            0 & 1 & \cdots & 0 & 0\\
            \vdots & \vdots & \ddots & \vdots & \vdots\\
            0 & 0 & \cdots & 1 & 0
        }\matr{
            a_{n-1}\\
            a_{n-2}\\
            \vdots\\
            a_{n-k+1}\\
            a_{n-k}
        }
        $$
        назовём первое $\vec{a_n}$, второе~--- $C$, третье~--- $\vec{a_{n-1}}$, всё, что нам надо, быстро возвести $C$ в степень.\\
        Но можно решить это за $k^2\log n$.
    \end{remark}
    \begin{example}
        Давайте возьмём числа Фибоначчи:
        $$
        f_n=f_{n-1}+f_{n-2}
        $$
        И мы знаем формулу
        $$
        \frac1{1-t-t^2}
        $$
        И домножим числитель и знаменатель на $1+t+t^2$:
        $$
        \frac{1+t+t^2}{1-3t^2+t^4}
        $$
        Если взять это и получить отсюда рекурренту, получим
        $$
        f_n=3f_{n-2}-f_{n-4}
        $$
        А если сделать это ещё раз, получим
        $$
        f_n=7f_{n-4}-f_{n-8}
        $$
        То есть нам нужно значительно меньше считать.
    \end{example}
    \begin{remark}
        Более глобально,
        $$
        A(t)=\frac{P(t)}{Q(t)}=\frac{P(t)Q(-t)}{Q(t)Q(-t)}=\frac{P_0(t^2)+P_1(t^2)t}{Q_1(t^2)}
        $$
        Таким образом можно решить задачку по поиску $a_n$ за $(m+k)k\log n$
    \end{remark}
    \begin{remark}
        Теперь давайте разбираться с третьим утверждением теоремы.
    \end{remark}
    \begin{lemma}
        Если $a_n=n^kr^n$, то $A$~--- дробно-рациональный формальный степенной ряд.
    \end{lemma}
    \begin{proof}
        Индукция по $k$. Для $r^n$ знаем ($\frac1{1-rt}$), переход: пусть $n^{k-1}r^n=\frac{P(t)}{Q(t)}$, то чему равно $n^{k}r^n$? Ну, возьмём производную и домножим на $t$. Получим, как нетрудно заметить, отношение многочленов.
    \end{proof}
    \begin{remark}
        На практике это не очень применимо т.к. очень быстро растёт степень знаменателя. Давайте докажем, что $n^kr^n$ представим как
        $$
        \frac{P_k(t)}{(1-rt)^{k+1}}
        $$
        Понятно, что $P_0=1$, и всё хорошо. Переход индукции:
        $$
        \left(P_k(t)\frac1{(1-rt)^k}\right)'t=P'_k(t)\frac t{(1-rt)^k}+P_k(t)\frac t{(1-rt)^{k+1}}
        $$
    \end{remark}
    \begin{proof}
        Мы уже доказали, что из 1 следует 3 в теореме. Докажем обратно.\\
        Воспользуемся основной теоремой алгебры. Пусть $\deg Q=k$, тогда у него $k$ корней с учётом кратности. Не умаляя общности $Q=1-c_1t-c_2t^2-\cdots-c_kt^k$. Пусть корни $Q$ равны $x_i$ и кратность у них $s_i$. Тогда $\frac{P(t)}{Q(t)}$, как мы знаем из математического анализа, можно разложить на простые дроби:
        $$
        \frac{P(t)}{Q(t)}=\sum\limits_{i=1}^l\frac{P_i(t)}{(1-r_it)^{s_i}}
        $$
        То есть нам достаточно доказать, что $\frac{t^m}{(1-rt)^{s}}$ представим как $a_n=p(n)r^n$. Давайте сначала разберёмся для $m=0$. И разберёмся индукцией по $s$. Для $s=1$ мы знаем, $p(n)\equiv1$. Что для больших $s$? Ну, продифференцируем нашу формулу
        $$B'(t)=\left(\frac1{(1-rt)^{s}}\right)'=\frac{sr}{(1-rt)^{s+1}}$$
        То есть $\frac1{(1-rt)^{s+1}}=\frac{B'(t)}{sr}$. И если раньше мы имели $p(n)r^n$, то теперь имеем $\frac{p(n+1)(n+1)r^{n+1}}{sr}$, и это то, что нам надо.\\
        А если брать $t^m$ в числителе, то получим $a_n=p(n-m)r^{-m}r^n$, если $m\leqslant n$.
    \end{proof}
    \begin{remark}
        Что, кстати, видим, так это то, что $\deg p=k-1$.
    \end{remark}
    \begin{remark}
        Также, как несложно заметить, асимптотика роста последовательности зависит только от наименьшего по модулю корня знаменателя. Остаётся лишь вопрос, что, если таковых несколько? Ну, рассмотрим
        $$
        \frac1{1-t^2}
        $$
        Если посмотреть на члены этой последовательности, то там чередуются нули и единицы. А в более общем случае, если взять
        $$
        M=\operatorname{lcm}\left(\frac{2\pi}{\arg r_i}\right)
        $$
        То последовательность распадётся на $M$ независимых, каждая из которых растёт с асимптотикой $u^n\left(e^{\im\varphi_i(n\%M)}\right)$, где $u$~--- модуль $r_i$.
    \end{remark}
    \paragraph{Использование производящих функций для работы с комбинаторными объектами.}
    \begin{remark}
        Давайте возьмём прямоугольник $2\times n$ и просуммируем все способы его замостить. И ещё для всех $n$ просуммируем.\\
        Все замещения, кроме пустого, начинаются либо с вертикальной доминошки, либо с двух горизонтальных. Давайте сгруппируем так слагаемые и вынесем за скобки начало. Нетрудно заметить, что в скобках все возможные замощения. Итого имеем, что сумма всех замощений $S$~--- пустое замощение плюс $S$ на сумму вертикальной доминошки и горизонтальной в квадрате. Умеем ли мы делить один на разность пустого замощения и суммы вертикальной доминошки и горизонтальной в квадрате? Ну, почему нет. Получится сумма всех замощений, это мы уже знаем.\\
        Но смотрите. Мы получили буквально формальные степенные ряды, если заменить количество доминошек на степень. И мы имеем, что фигура из $k$ доминошек имеет <<вес>> $d^k$, и по сути мы тут считаем комбинаторыне объекты заданного веса.\\
        То же самое можно делать и для подвешенных двоичных деревьев, для любых комбинаторных объектов по сути. Но возникает вопрос: зачем? А на самом деле, тут мы используем производящие функции для работы с комбинаторными объектами.
    \end{remark}
    \begin{remark}
        Что у нас по сути есть? У нас есть некоторая неделимая штука~--- <<атом>>. И мы определённым образом производим действия с этими атомами, после чего получается объект с каким-то количеством атомов или <<весом>>. А интересно нам в этой теории количество комбинаторных объектов заданного веса. Ну, эти количества задают какую-то последовательность, а значит можно сделать производящую функцию.\\
        И как же эту функцию получить? Тут нам как раз поможет формальная сумма объектов. Если их просуммировать, а потом взять расчленить каждый на атомы, получится сумма $u^{n_k}$. И после приведения подобных как раз получится формальный степенной ряд.
    \end{remark}
    \begin{definition}
        \textbf{Дизъюнктным объединением} комбинаторных объектов $A$ и $B$ называется комбинаторный объект, производящая функция которого равна сумме формальных степенных рядов $A$ и $B$.
    \end{definition}
    \begin{remark}
        Почему? Потому что если у нас есть два множества комбинаторных объектов не пересекаются, и объединение имеет своим весом сумму весов эти множеств.
    \end{remark}
    \begin{definition}
        \textbf{Парой} комбинаторных объектов $A$ и $B$ называется комбинаторный объект с производящей функцией, равным произведению производящих функций $A$ и $B$.
    \end{definition}
    \begin{remark}
        Опять же, почему? Ну, если считать вес пары равным сумме весов её компонент, то получится именно что
        $$c_n=\sum\limits_{i=0}^na_ib_{n-i}$$
    \end{remark}
    \begin{remark}
        Теперь давайте сделаем последовательность комбинаторных объектов $A$. Во-первых, если в $A$ есть объект веса 0, его можно вставить куда угодно в список в любом количестве, а значит объектов любого веса будет бесконечное количество. Плохо.\\
        Пусть не так. Тогда что будет? Вспомним Clojure и скажем, что у нас есть пустой список (он один), а если список не пуст, то у него есть хвост и голова. Голова~--- это $A$, хвост~--- это $\Seq A$. Итого
        $$
        \Seq A=1+A\times\Seq A\Leftrightarrow\Seq A=\frac1{1-A}
        $$
    \end{remark}
    \begin{definition}
        Пусть $A$~--- комбинаторный объект. Тогда $\Seq A$~--- комбинаторный объект, производящая функция которого задаётся как
        $$
        \frac1{1-A}
        $$ 
    \end{definition}
    \begin{example}
        Начинаем веселиться. Пусть у нас есть не один атом, а два. Разных. Ну, ничего интересного, $B=\{u_1,u_2\}$ имеет производящую функцию $B(t)=2t$. И дальше ничего концептуально нового. Можно рассмотреть $\Seq B$ и получить $\frac1{1-2t}$.
    \end{example}
    \begin{example}
        Можно рассмотреть что-нибудь концептуально ещё более сложное, например, $A=\{u;(u;u)\}$. Это $t+t^2$. Тогда $\Seq A=\frac1{1-t-t^2}$. Откуда же тут числа Фибоначчи? Да понятно, откуда, у нас либо вертикальная доминошка ($u$), либо пара горизонтальных $(u;u)$. Это ровно тот пример, что мы рассматривали изначально.
    \end{example}
    \begin{remark}
        Дальше мы хотим множество всех подмножеств. Понятно, что это, но непонятно, как возвести $2$ в степень ряда. Ну, смотрите. Рассмотрим наши комбинаторные объекты $A$. Это какие-то $A_1,A_2$ и так далее. Мы можем взять или не взять первый элемент. Это $1+A_1$. Можем взять или не взять $A_2$: $1+A_2$. Итого
        $$
        \prod\limits_{a\in A}(1+a)=\prod\limits_{a\in A}(1+t^{w(a)})=\prod\limits_{k=0}^\infty(1+t^k)^{a_k}
        $$
    \end{remark}
    \begin{definition}
        Множеством всех подмножеств (powerset) комбинаторного объекта $A(t)=a_0+a_1t+\cdots$ называется комбинаторный объект, производящая функция которого равна
        $$
        \prod\limits_{k=0}^\infty(1+t^k)^{a_k}
        $$
    \end{definition}
    \begin{remark}
        Теперь хочется мульти-множество. Мы можем взять каждый объект несколько раз. Тут, очевидно, тоже не может быть объектов веса ноль. Можем взять первый объект любое количество раз, второй~--- тоже и так далее. Получается
        $$
        \left(1+A_1+A_1^2+\cdots\right)\left(1+A_2+A_2^2+\cdots\right)\cdots
        $$
        Что в итоге получается?
        $$
        \frac1{1-A_1}\frac1{1-A_2}\cdots
        $$
    \end{remark}
    \begin{definition}
        Мульти-множеством (multiset) комбинаторного объекта $A(t)=a_0+a_1t+\cdots$ называется комбинаторный объект, производящая функция которого равна
        $$
        \prod\limits_{k=1}^\infty\left(\frac1{1-t^k}\right)^{a_k}
        $$
    \end{definition}
    \begin{claim}
        $$
        \MSet A=\PSet(\Seq A-1)
        $$
    \end{claim}
    \begin{remark}
        Пусть $A$~--- комбинаторный объект. Давайте делать циклы из $A$. Как устроен цикл? Как список, но с точностью до циклического сдвига.\\
        Мы знаем, что $\Seq A=1+A+A^2+\cdots$. Понятно, что $A^k$ не зависят друг от друга, значит можно рассмотреть $A^k$ отдельно. Рассмотрим. Это кортеж из $k$ элементов $A$. Воспользуемся леммой Бернсайда. Пусть $I_{n,k,i}$~--- количество массивов длины $k$ веса $n$, в которых $i$~--- неподвижная точка. Тогда
        $$
        C_{k,n}=\frac1k\sum\limits_{i=0}^{k-1}I_{n,k,i}
        $$
        $$
        I_{n,k,i}=\begin{cases}
            0 & n\not\divby\frac k{\gcd(k,i)}\\
            A^{\gcd(k,i)}_{\frac{n\gcd(k,i)}k} &
        \end{cases}
        $$
        Тогда $C_n$~--- это сумма $C_{n,k}$. Говорят, что все эти три штуки можно упростить до
        $$
        C(t)=\sum\limits_{k=1}^\infty\frac{\varphi(k)}k\ln\frac1{1-A(t^k)}
        $$
        Где $\varphi$~--- функция Эйлера.
    \end{remark}
    \begin{remark}
        Здесь дыра в сюжете.
    \end{remark}
    \paragraph{Производящие функции Дирихле.}
    \begin{remark}
        Сегодня в качестве формальной переменной мы будем использовать $s$, а не $t$
    \end{remark}
    \begin{definition}
        \textbf{Производящая функция Дирихле} для последовательности $a_1,a_2,\ldots$~--- это следующий формальный ряд
        $$
        A(s)=\sum\limits_{n=1}^\infty \frac{a_n}{n^s}
        $$
    \end{definition}
    \begin{example}
        Начнём с последовательности $1,0,0,\ldots,0,\ldots$. Для неё $A(s)=1$.
    \end{example}
    \begin{example}
        $1,1,\ldots,1,\ldots$. Тогда $A(s)=\zeta(s)$~--- дзета-функция Римана.
    \end{example}
    \begin{example}
        $1,2,3,4,\ldots$. Тогда $A(s)=\zeta(s-1)$.
    \end{example}
    \begin{claim}
        Если $k\in\mathbb Z,b_n=n^ka_n$, то $B(s)=A(s-k)$.
    \end{claim}
    \begin{claim}
        Несложно заметить, что сумма производящих функций Дирихле~--- производящая функция Дирихле суммы рядов.
    \end{claim}
    \begin{remark}
        Отсюда мы легко умеем умножать производящие функции Дирихле на многочлен
    \end{remark}
    \begin{remark}
        Пусть $C(s)=A(s)B(s)$, то как $c_n$ зависит от $a_n$ и $b_n$?
        $$
        C(s)=\left(\sum\limits_{n=1}^\infty\frac{a_n}{n^s}\right)\left(\sum\limits_{k=1}^\infty\frac{b_n}{n^s}\right)=\sum\limits_{n=1}^\infty\sum\limits_{k=1}^\infty\frac{a_nb_k}{(nk)^s}=\sum\limits_{n=1}^\infty\sum\limits_{m\mathrel{|}n}\frac{a_nb_{m/n}}{n^s}
        $$
    \end{remark}
    \begin{corollary}
        Отсюда если $A(s)\zeta(s)=B(s)$, то $b_n=\sum\limits_{d\mathrel{|}n}a_d$.
    \end{corollary}
    \begin{corollary}
        Отсюда последовательность $d_n$~--- количество делителей $n$ соотвествует производящей функции $\zeta^2(s)$.
    \end{corollary}
    \begin{remark}
        Давайте научимся делить. Попробуем взять $\frac1{\zeta(s)}$. Обозначим это за $\mu(s)$.
        $$
        \sum\limits_{d\mathrel|1}\mu_d=1\Rightarrow\mu_1=1
        $$
        $$
        \sum\limits_{d\mathrel|2}\mu_d=0\Rightarrow\mu_2=-1
        $$
        $$
        \sum\limits_{d\mathrel|3}\mu_d=0\Rightarrow\mu_3=-1
        $$
        $$
        \sum\limits_{d\mathrel|4}\mu_d=0\Rightarrow\mu_4=0
        $$
        Заметим, что это единственным образом определяет $\mu$. Мы получим $\mu_5=-1,\mu_6=1$. Для простых мы видим, что оно $-1$. А для составных непонятно.
    \end{remark}
    \begin{theorem}
        Пусть $\mu(s)=\frac1{\zeta(s)}$. Тогда
        $$
        \mu_k=\begin{cases}
            0 & \exists p~p^2\mathrel|k\\
            (-1)^\nu & k=p_1p_2\cdots p_\nu,p_i\text{ различны}
        \end{cases}
        $$
    \end{theorem}
    \begin{proof}
        Докажем, что $\mu$, определённое той формулой подходит под $\zeta(s)\mu(s)=1$. В $n=1$ понятно, пусть $n>1$. Хочется
        $$
        0=\sum\limits_{d\mathrel|n}\mu_d
        $$
        Пусть $n=p_1^{\alpha_1}p_2^{\alpha_2}\cdots p_k^{\alpha_k}$. Если квадрат есть, понятно, иначе пусть мы выбрали $j$ простых множителей. Сколько способов их выбрать? $\Cnk kj$. Получим
        $$
        \sum\limits_{d\mathrel|n}\mu_d=\sum\limits_{j=0}^k\Cnk kj(-1)^j=(1-1)^k=0
        $$
    \end{proof}
    \begin{definition}
        Функция $\mu(s)$ называется \textbf{функцией Мёбиуса}.
    \end{definition}
    \begin{claim}[Формула обращения Мёбиуса]
        Пусть $B(s)=A(s)\zeta(s)$. Чтобы найти $A$, можно домножить обе части равенства на $\mu$ и получить
        $$
        a_n=\sum\limits_{d\mathrel|n}b_d\mu_{n/d}
        $$
    \end{claim}
    \begin{definition}
        Произведение $\prod\limits_{p\in\mathbb P}\sum\limits_{k=0}^\infty\frac1{p^{ks}}$ называется \textbf{произведением Эйлера}.
    \end{definition}
    \begin{theorem}
        $$
        \prod\limits_{p\in\mathbb P}\sum\limits_{k=0}^\infty\frac1{p^{ks}}=\zeta(s)
        $$
    \end{theorem}
    \begin{proof}
        Что это за штука? Ну,
        $$
        \left(1+\frac1{2^s}+\frac1{4^s}+\cdots\right)\left(1+\frac1{3^s}+\frac1{9^s}+\cdots\right)\left(1+\frac1{5^s}+\frac1{25^s}+\cdots\right)\cdots
        $$
        Ну, что у нас будет в начале? Ну, $1$. Потом мы получим $\frac1{2^s}$, $\frac1{3^s}$ и $\frac1{5^s}$. Если взять несколько не-единиц, будет $\frac1{6^s}$, $\frac1{10^s}$, $\frac1{15^s}$. Ну, о'кей, а зачем всё это? А затем, что мы получим любое $\frac1{n^s}$, притом ровно один раз, потому что $n$ единственным образом можно разложить на простые и взять единственным образом эти простые из скобок.
    \end{proof}
    \begin{remark}
        Давайте вот на что посмотрим:
        $$
        \prod\limits_{p\in\mathbb P}\sum\limits_{k=0}^\infty\frac1{p^{ks}}=\prod\limits_{p\in\mathbb P}\sum\limits_{k=0}^\infty\frac1{(p^s)^k}
        $$
        Это геометрическая прогрессия! Она равна
        $$
        \prod\limits_{p\in\mathbb P}\sum\limits_{k=0}^\infty\frac1{(p^s)^k}=\prod\limits_{p\in\mathbb P}\frac1{1-p^{-s}}
        $$
        На что надо умножить это, чтобы получилась единица?
    \end{remark}
    \begin{corollary}
        $$
        \mu(s)=\prod\limits_{p\in\mathbb P}\left(1-\frac1{p^s}\right)
        $$
    \end{corollary}
    \begin{definition}
        Будем обозначать взаимно простые числа значком $\perp$.
    \end{definition}
    \begin{definition}
        Рассмотрим $f\colon\mathbb N\to\mathbb R$ \textbf{теоретико-численно мультипликативной} (далее просто мультипликативной), если
        $$
        \forall u,v:u\perp v~f(uv)=f(u)f(v)
        $$
        Последовательность называется мультипликативной, если она мультипликативна как функция индексов.
    \end{definition}
    \begin{example}
        $f(u)=1$, хорошая функция.
    \end{example}
    \begin{remark}
        Пусть $a_n$~--- мультипликативная последовательность. Пусть $n=p_1^{\alpha_1}p_2^{\alpha_2}\cdots$. Тогда несложно заметить, что достаточно узнать $a_{p_i^k}$.
    \end{remark}
    \begin{theorem}
        Если $a_n$ мультипликативно, то
        $$
        A(s)=\prod\limits_{p\in\mathbb P}\sum\limits_{k=0}^\infty\frac{a_{p^k}}{p^{ks}}
        $$
    \end{theorem}
    \begin{definition}
        $\sigma_k(s)=\sum\limits_{d\mathrel|n}d^k$ называется \textbf{теоретико-числовыми моментами числа}.
    \end{definition}
    \begin{example}
        $\sigma_0(n)$~--- количество делителей числа $n$. Чему равно $\Sigma_0(s)$?
        $$
        \Sigma_0(s)=\prod\limits_{p\in\mathbb P}\sum\limits_{k=0}^\infty\frac{k+1}{p^{ks}}=
        \prod\limits_{p\in\mathbb P}\underbrace{\sum\limits_{k=0}^\infty\frac{k+1}{(p^s)^k}}_{\frac1{(1-p^{-s})^2}}=
        \prod\limits_{p\in\mathbb P}\frac1{(1-p^{-s})^2}=\zeta(s)^2
        $$
    \end{example}
    \begin{example}
        Рассмотрим функцию Эйлера $\phi(n)$~--- количество натуральных чисел, взаимно простых с $n$ и меньших его.
        \[\begin{split}
            \Phi(s)&=\prod\limits_{p\in\mathbb P}\left(1+\sum\limits_{k=1}^\infty\frac{p^k-p^{k-1}}{p^{ks}}\right)=\prod\limits_{p\in\mathbb P}\left(1+\frac{p^1-p^0}{p^s}+\frac{p^2-p^1}{p^{2s}}+\cdots\right)=\\
            &=\prod\limits_{p\in\mathbb P}\left(\underbrace{1+\frac{p^1}{p^s}+\frac{p^2}{p^{2s}}+\cdots}_{\sum\limits_{k=0}^\infty\frac1{p^{k(s-1)}}}+\underbrace{-1-\frac{p^0}{p^s}-\frac{p^1}{p^{2s}}-\cdots}_{-\frac1p\sum\limits_{k=1}^\infty\frac1{p^{k(s-1)}}}\right)=\\
            &=\prod\limits_{p\in\mathbb P}\frac{1-p^{-s}}{1-p^{-s+1}}=\mu(s)\zeta(s-1)=\frac{\zeta(s-1)}{\zeta(s)}
        \end{split}\]
    \end{example}
    \section{Теория вычислимости.}
    \paragraph{Общие понятия и утверждения.}
    \begin{remark}
        Глобально мы хотим анализировать как-то алгоритмы. А для этого надо формализовать понятие <<программы>> и <<алгоритма>>. Ну, это достаточно просто. У нас есть какие-то символы $\Pi$ и некоторое подмножество строк из них~--- корректные программы. Какие~--- пока не так важно. А зафиксируем то, что программа принимает на вход какой-то $x$~--- строку символов из $\Sigma$ и возвращает что? Ну, мы для простоты будем возвращать истину или ложь. Этого нам хватит. Если программа крашится~--- будем считать, что это ложь.\\
        Но ведь программа может никогда не завершиться. И это уже какой-то другой фундаментальный случай, его нельзя проэмулировать и узнать.
    \end{remark}
    \begin{definition}
        Будем считать, что \textbf{язык программы} ($L(p)$)~--- множество слов, на которых она не зависает и выдаёт 1.
    \end{definition}
    \begin{claim}
        Несложно заметить, что $L(p)$~--- функция из программ в языки. И что она не инъективна (\Verb|return false;| и \Verb|while (true) {}| дают пустой язык, хотя программы разные). И что по соображениям мощности эта функция не сюръективна. Более того, $2^{\Sigma^*}$ (все языки)~--- несчётное множество, а множество программ счётно.
    \end{claim}
    \begin{remark}
        Впрочем, как мы обсуждали ранее, у невозможности задания почти всех языков программой есть и обратная сторона. Очень трудно описать несчётное количество языков счётным количеством строк на русском языке.
    \end{remark}
    \begin{definition}
        Язык называется \textbf{рекурсивно-перечислимым}, если он распознаётся некоторой программой. Множество таких языков~--- $RE$. Более современное название~--- просто \textbf{перечислимые} или \textbf{полуразрешимые}.
    \end{definition}
    \begin{remark}
        Почему \textbf{полу}-разрешимые? Потому что программа умеет зависать, то есть мы нормально живём только в том случае, если слово есть в языке.
    \end{remark}
    \begin{definition}
        Язык называется \textbf{рекурсивным} (устаревшее) или \textbf{разрешимым} (современное), если существует программа, которая никогда не зависает и возвращает истину только на словах этого языка.
    \end{definition}
    \begin{remark}
        Так. У нас есть два алфавита. Зачем? Давайте всё будет в одном алфавите. Компьютер нормально живёт с битовыми строками, и всё у него хорошо. Так что отныне $\Sigma=\Pi$.
    \end{remark}
    \begin{remark}
        Ещё одно допущение. Зачем нам считать, что не любая программа корректна? Давайте любая программа будет корректна, а если нет, то подставим вместо него программу, которая всегда зависает.
    \end{remark}
    \begin{remark}
        Так, хорошо. Ещё иногда удобно считать, что программа принимает не строку, а неотрицательное целое число. Между $\mathbb N_0$ и $\Sigma^*$ можно создать простую биекцию: градуированный лексикографический порядок.
    \end{remark}
    \begin{definition}
        \textbf{Градуированный лексикографический порядок}~--- упорядочивание сначала по длине, а среди равных длин~--- лексикографически. Он (как биекция между $\Sigma^*$ и $\mathbb N_0$) называется \textbf{естественным изоморфизмом}.
    \end{definition}
    \begin{remark}
        Здесь и далее $\mathbb N_0$, $\Sigma^*$ и множество программ можно использовать рандомно, они все изоморфны, а значит можно говорить о программах, как о числах, о вводе как и программе и т.п.
    \end{remark}
    \begin{definition}
        Номер программы в естественном изоморфизме~--- \textbf{номер Гёделя}.
    \end{definition}
    \begin{remark}
        Хотя вообще номер Гёделя~--- что-то более комплексное т.к. более глобально нам хочется сделать программу, которая считает число, конвертирует его в программу и запустит её. И вот в \textbf{любой} такой нумерации номера называются номерами Гёделя. Наш пример~--- именно такая нумерация.
    \end{remark}
    \begin{remark}
        Всё это, конечно, хорошо, но что такое программа? А на самом деле не так важно. Есть два подхода к этому. Можно формально определить, что такое программа, углубиться в абстрактные вычислители и прочее. А второе~--- программистский здравый смысл. Мы с вами программисты, знаем кучу всего вообще, давайте считать, что у нас есть языки программирования есть, и нам разве что пофиг, что у нас ограниченная память, ограниченные целые числа и т.п. Но на что-то забить всё же нельзя. На точность вещественных чисел, например, иначе парадоксы получатся.\\
        Очень хорошо, а верно ли, что наши два подхода эквивалентны? Ну, вообще вопрос некорректный, потому что слева математический объект, а справа хрень какая-то, но если и на это забить, то да, эквивалентны. Это \textbf{тезис Тьюринга~--- Чёрча}. По модулю ограниченности памяти наши штуки эквивалентны.
    \end{remark}
    \begin{remark}
        Почему полуразрешимый язык называется перечислимым?
    \end{remark}
    \begin{definition}
        Язык называется \textbf{перечислимым}, если существует программа, которая, будучи запущенной на пустов вводе выводит все слова языка (возможно, за бесконечное время).
    \end{definition}
    \begin{remark}
        Тут что-то схожее с производящими функциями. Мы не можем описать заданный член формального ряда, но можем описать первые $n$, когда они заданы рекуррентно. Ещё есть вопрос в том, что значит <<выводит>>, как оно их разделяет, но это всё не важно.
    \end{remark}
    \begin{example}
        Десятичные записи простых чисел перечислимы. Понятно, как их перечислить. Напишем простую проверку, является ли число простым и просто в \Verb|for| запихнём её вызов и печать.
    \end{example}
    \begin{theorem}
        Перечислимость и полуразрешимость эквивалентны.
    \end{theorem}
    \begin{proof}
        Давайте полуразрешим перечислимый язык. Давайте возьмём перечислитель и всё, что он напечатает, перехватим и будем поочерёдно сравнивать с пришедшим нам словом.\\
        Давайте наоборот. Заметим, что перебрать все слова и проверить каждое не получится (проверка может зависнуть). Давайте запускать программу с ограничением на время работы и сначала переберём все слова с ограничением на время работы $TL=1$, потом $TL=2$ и т.д. Правда, тут вложенные бесконечные циклы, и в итоге до $TL=2$ мы не доберёмся. Поэтому давайте напишем вот что:
        \begin{verbatim}
            for TL in range(1, infinity):
                for x in map(word_by_number, range(TL)):
                    if P(x):
                        print(x)
        \end{verbatim}
        Что мы тут использовали? Что мы можем вставить одну программу в другую, причём можно ещё и ограничить каким-то количеством шагов.
    \end{proof}
    \begin{theorem}
       Любой разрешимый язык полуразрешим, но не обратное.
    \end{theorem}
    \begin{example}
        Рассмотрим вот такой язык: пусть $U=\{(p;x)\mid p(x)=1\}$~--- множество пар из программы и ввода, что ввод удовлетворяет программе. Понятно, что он полуразрешим (запустим $p(x)$). Но почему он не разрешим? От противного, пусть есть программа $q$, которая его разрешает. Сделаем тогда программу $r(x)=\neg q(x;x)$. Эта штука также никогда не зависает. Чудно. Что такое $r(r)$? Если это 1, то $q(r;r)=0\Rightarrow u(r;r)\neq1\Rightarrow r(r)\neq1$. Если 0, то $q(r;r)=1$, а значит $r(r)=1$. Ой.
    \end{example}
    \begin{theorem}
        Если $A$ и $\overline A$ перечислимы, то $A$ разрешим.
    \end{theorem}
    \begin{proof}
        Ну, один из этих двоих не зависнет. А значит вводим $TL$ и ждём того, кто первый пройдёт. В программировании можно воспользоваться тезисом Тьюринга~--- Чёрча и запустить два потока исполнения.
    \end{proof}
    \begin{claim}
        Существуют неперечислимые языки.
    \end{claim}
    \begin{proof}
        Возьмём $A=U$~--- перечислим, но не разрешим. Тогда $\overline U$ по теореме выше неперечислим.
    \end{proof}
    \begin{remark}
        А можно ли, чтобы программы не зависали? Ну, мы ещё выясним, что нельзя выполнить три условия: программы не зависают, не-зависающие программы эквивалентны обычными и чтобы по программе можно было понять, что она выводит.
    \end{remark}
    \begin{definition}
        Рассмотрим последовательность программ $\{p_n\}$. Говорят, что \textbf{последовательность программ вычислима}, если существует программа $V(i;x)$, которая по номеру $i$ и входу $x$ возвращает $p_i(x)$.
    \end{definition}
    \begin{theorem}
        Пусть $\{p_n\}$~--- последовательность программ. Тогда выполнено хотя бы одно из трёх:
        \begin{enumerate}
            \item Существуют $i$ и $x$ такие что $p_i(x)$ зависает.
            \item Существует такая программу $q$, что она не зависает ни на одном входе и для любого $i$ существует $x_i$ такое что $q(x_i)\neq p_i(x_i)$
            \item $\{p_i\}$ не вычислима.
        \end{enumerate}
    \end{theorem}
    \begin{proof}
        Пусть не так. Рассмотрим $q(x)=\neg V(x;x)$ (поскольку $\{p_i\}$ вычислима). Тогда, поскольку $p_i$ не зависают, $q$ она подходит под 2.
    \end{proof}
    \begin{claim}[Задача останова]
        $HALT=\{(p;x)\mid p\text{ не зависает на }x\}$ не разрешим.
    \end{claim}
    \begin{proof}
        Ну, пусть мы разрешили задачу останова программой $h$. Пусть $q(x)$~--- программа, которая выглядит так:
        \begin{verbatim}
            if (h(x, x))
                while (true)
                    ;
            else
                return 0;
        \end{verbatim}
        Понятно, что попытка проанализировать $(q;q)$ приведёт нас к противоречию.
    \end{proof}
    \begin{definition}
        \textbf{Функция} $f\colon\underset{\subset\Sigma^*}A\to\Sigma^*$ называется \textbf{вычислимой}, если существует такая программа $p$, что $\forall x\in A~f(x)=p(x)$ и $\forall x\notin A~p$ зависает.
    \end{definition}
    \begin{definition}
        Всюду определённая функция~--- та, у которой область определения равно $\Sigma^*$.
    \end{definition}
    \begin{definition}
        Говорят, что $A$ \textbf{m-сводится} к $B$, если существует всюду определённая вычислимая функция $f$ такая что $x\in A\Leftrightarrow f(x)\in B$. Обозначение: $A\leqslant_mB$
    \end{definition}
    \begin{remark}
        То есть все слова из $A$ отображаются в язык $B$, а не из $A$~--- вне $B$. При этом функция нигде не зависает.
    \end{remark}
    \begin{remark}
        <<m>> исторически расшифровывается как many-to-one, а современно как mapping.
    \end{remark}
    \begin{theorem}
        Если $B$ разрешим и $A\leqslant_m B$, то $A$ разрешим.
    \end{theorem}
    \begin{proof}
        Ну, тривиально, нам нужна композиция функции и разрешителя $B$.
    \end{proof}
    \begin{theorem}
        Если $B$ перечислим и $A\leqslant_m B$, то $A$ перечислим.
    \end{theorem}
    \begin{theorem}
        Если $A$ не разрешим/не перечислим, $A\leqslant_m B$, то $B$ не разрешим/не перечислим.
    \end{theorem}
    \begin{theorem}
        $U\leqslant_m HALT$.
    \end{theorem}
    \begin{proof}
        Пусть $f$ преобразует программу $p$ и ввод $x$ в
        \begin{verbatim}
            q = function(x)
                return "if (p(" + x + ") != 1)
                    while (true)
                        ;
                else
                    return 1"
            return (q, x)
        \end{verbatim}
    \end{proof}
    \begin{claim}
        Пусть $HALT_\varepsilon$~--- множество программ, не зависающих на $\varepsilon$. Тогда
        $$
        HALT\leqslant_mHALT_\varepsilon
        $$
    \end{claim}
    \begin{proof}
        Программа сведения выглядит так:
        \begin{verbatim}
            return "function(unused) p(" + x + "), return 0"
        \end{verbatim}
    \end{proof}
    \begin{definition}
        $FINITE=\{p\mid\exists\text{конечное число }x~p(x)=1\}$
    \end{definition}
    \begin{claim}
        $HALT\leqslant_m\overline{FINITE}$.
    \end{claim}
    \begin{theorem}[Теорема Успенского~--- Райса]
        Никакое нетривиальное свойство перечислимых языков не разрешимо.
    \end{theorem}
    \begin{definition}
        $A\subset RE$ ($RE$, напомню, множество полуразрешимых языков) называется \textbf{свойством}.
    \end{definition}
    \begin{definition}
        Множество программ, распознающих языки с заданным свойством~--- \textbf{язык свойства}.
    \end{definition}
    \begin{definition}
        Отсюда мы сразу понимаем, что такое \textbf{разрешимое свойство}~--- то, язык которого разрешим.
    \end{definition}
    \begin{definition}
        \textbf{Свойство тривиально}, если оно равно $RE$ или $\varnothing$.
    \end{definition}
    \begin{proof}
        Пусть $A$~--- нетривиальное свойство. То есть есть язык, который ему принадлежит, а есть тот, который не. Возьмём пустой язык. Не умаляя общности, пусть он не принадлежит $A$. И есть $T\in A$. У $T$ есть полуразрешитель \Verb|inT|. Пусть $A$ разрешим. Тогда есть программа \Verb|inA(p)|, которая проверяет, верно ли, что $L(p)\in A$. Напишем такую программу:
        \begin{verbatim}
            inU(p, x):
                q = "function(y)
                        if (p(" + x + ") == 1)
                            return inT(y)
                        else
                            return 0"
                return inA(q)
        \end{verbatim}
        Утверждается, что это разрешитель универсального языка. Если $p(x)=1$, то $q(y)\equiv inT(y)$. В противном случае $q(y)=0$. Если $p(x)$ зависает, то $q(y)$ тоже зависает. Итого, если $p(x)=1$, то $q$ полураспознаёт $T$, а значит $L(q)=T\in A$. В противном случае $q$ полураспознаёт пустой язык, $L(q)=\varnothing\notin A$. То есть $p(x)=1\Leftrightarrow L(q)\in A\Leftrightarrow inA(q)=1$.
    \end{proof}
    \begin{remark}
        Здесь очень важное слово~--- <<языков>>. Не <<программ>>. Если вы прикручиваете свойство к программе, то теорема перестаёт работать.
    \end{remark}
    \subparagraph{Квайны.}
    \begin{remark}
        Сегодня на повестке дня <<как написать программу, которая выводит свой собственный код, и зачем>>.\\
        напишем простую программу, которая печатает ничего:
        \begin{verbatim}
            #include <iostream>
            
            int main() {
                std::cout << "";
                return 0;
            }
        \end{verbatim}
        Идея первая~--- скопировать этот код внутрь кавычек
        \begin{verbatim}
            #include <iostream>
            
            int main() {
                std::cout << "#include <iostream>
                
                int main() {
                    std::cout << "";
                    return 0;
                }";
                return 0;
            }
        \end{verbatim}
        Но он помимо к тому, что не работает, ещё и не компилируется. Ладненько, давайте поборемся со вторым, научимся экранировать:
        \begin{verbatim}
            std::string escape(const std::string& s) {
                std::string result;
                for (char c : s) {
                    switch (c) {
                    case '\n':
                        result += "\\n";
                        break;
                    case '\"':
                        result += "\\\"";
                        break;
                    case '\'':
                        result += "\\'";
                        break;
                    case '\\':
                        result += "\\\\";
                        break;
                    default:
                        result += c;
                        break;
                    }
                }
                return result;
            }
        \end{verbatim}
        Давайте через эту штуку пропустим нашу программу. Получим что-то такое:
        \begin{verbatim}
            #include <iostream>\n#include <string>\n\nstd::string escape(const std::string& s) {\n    std::string result;\n    for (char c : s) {\n        switch (c) {\n        case \'\\n\':\n            result += \"\\\\n\";\n            break;\n        case \'\\\"\':\n            result += \"\\\\\\\"\";\n            break;\n        case \'\\\'\':\n            result += \"\\\\\'\";\n            break;\n        case \'\\\\\':\n            result += \"\\\\\\\\\";\n            break;\n        default:\n            result += c;\n            break;\n        }\n    }\n    return result;\n}\n\nint main() {\n    std::cout << \"\";\n    return 0;\n}\n
        \end{verbatim}
        Теперь просто засунем это на место строки в исходный код:
        \begin{verbatim}
            #include <iostream>

            std::string escape(const std::string& s) {
                std::string result;
                for (char c : s) {
                    switch (c) {
                    case '\n':
                        result += "\\n";
                        break;
                    case '\"':
                        result += "\\\"";
                        break;
                    case '\'':
                        result += "\\'";
                        break;
                    case '\\':
                        result += "\\\\";
                        break;
                    default:
                        result += c;
                        break;
                    }
                }
                return result;
            }

            int main() {
                std::cout << "#include <iostream>\n#include <string>\n\nstd::string escape(const std::string& s) {\n    std::string result;\n    for (char c : s) {\n        switch (c) {\n        case \'\\n\':\n            result += \"\\\\n\";\n            break;\n        case \'\\\"\':\n            result += \"\\\\\\\"\";\n            break;\n        case \'\\\'\':\n            result += \"\\\\\'\";\n            break;\n        case \'\\\\\':\n            result += \"\\\\\\\\\";\n            break;\n        default:\n            result += c;\n            break;\n        }\n    }\n    return result;\n}\n\nint main() {\n    std::cout << \"\";\n    return 0;\n}\n";
                return 0;
            }
        \end{verbatim}
        Ну, отличный план, только в том, что мы печатаем написано \Verb|std::cout << "";|, а у нас на месте этого какая-то упоротая строка. Но как ни странно, эта проблема решается. В нашей текущей программе уже написано то, что мы должны вставить вместо \Verb|std::cout << "";|. Давайте наша программа просто сделает то, что сделали мы: возьмёт этот исходный код, пропустит его через \Verb|escape| и подставит его на место \Verb|std::cout << "";|.\\
        Обозначим за \Verb|$| особый спецсимвол, которым мы покажем, где у нас находится \Verb|std::cout << "";|, чтобы знать, куда вставлять. Напишем пока что-то такое:
        \begin{verbatim}
            #include <iostream>
            #include <string>
            
            std::string escape(const std::string& s) {
                std::string result;
                for (char c : s) {
                    switch (c) {
                    case '\n':
                        result += "\\n";
                        break;
                    case '\"':
                        result += "\\\"";
                        break;
                    case '\'':
                        result += "\\'";
                        break;
                    case '\\\':
                        result += "\\\\";
                        break;
                    default:
                        result += c;
                        break;
                    }
                }
                return result;
            }
            
            std::string get_other_source() {
                return "$";
            }
            
            std::string get_source() {
                std::string s = get_other_source();
                std::string r = escape(s);
                s.replace(s.find('$'), 1, r);
                return s;
            }
            
            int main() {
                std::cout << get_source();
                return 0;
            }
        \end{verbatim}
        Теперь пропустим нашу программу через \Verb|escape| и вместо доллара поставим результат:
        \begin{verbatim}
            #include <iostream>
            #include <string>
            
            std::string escape(const std::string& s) {
                std::string result;
                for (char c : s) {
                    switch (c) {
                    case '\n':
                        result += "\\n";
                        break;
                    case '\"':
                        result += "\\\"";
                        break;
                    case '\'':
                        result += "\\'";
                        break;
                    case '\\':
                        result += "\\\\";
                        break;
                    default:
                        result += c;
                        break;
                    }
                }
                return result;
            }
            
            std::string get_other_source() {
                return "#include <iostream>\n#include <string>\n\nstd::string escape(const std::string& s) {\n    std::string result;\n    for (char c : s) {\n        switch (c) {\n        case \'\\n\':\n            result += \"\\\\n\";\n            break;\n        case \'\\\"\':\n            result += \"\\\\\\\"\";\n            break;\n        case \'\\\'\':\n            result += \"\\\\\'\";\n            break;\n        case \'\\\\\':\n            result += \"\\\\\\\\\";\n            break;\n        default:\n            result += c;\n            break;\n        }\n    }\n    return result;\n}\n\nstd::string get_other_source() {\n    return \"$\";\n}\n\nstd::string get_source() {\n    std::string s = get_other_source();\n    std::string r = escape(s);\n    s.replace(s.find(\'$\'), 1, r);\n    return s;\n}\n\nint main() {\n    std::cout << get_source();\n    return 0;\n}\n";
            }
            
            std::string get_source() {
                std::string s = get_other_source();
                std::string r = escape(s);
                s.replace(s.find('$'), 1, r);
                return s;
            }
            
            int main() {
                std::cout << get_source();
                return 0;
            }
        \end{verbatim}
        Оно работает!\\
        Заметим, что нам не надо именно выводить код. Мы можем, например, посчитать количество символов в исходном коде или что-нибудь ещё. То есть схема работы всегда одинаковая. Пишем \Verb|escape|, \Verb|get_other_source| и \Verb|source|, они одинаковые везде. После этого пишем в \Verb|main|'е то, что мы хотим сделать с исходным кодом. Когда мы кончили, пропускаем код программы через \Verb|escape| и вставляем вместо доллара.
    \end{remark}
    \begin{theorem}[Теорема о рекурсии, теорема о рефлексии]
        Для любой вычислимой функции двух аргументов $V(x;y)$ существует вычислимая функция одного аргумента $p$, что $\forall y~V(p;y)=p(y)$.
    \end{theorem}
    \begin{remark}
        Что происходит? А вот что: программа может использовать свой исходный код (любым образом).\\
        Доказывать эту теорему мы не будем, а дальше рассмотрим применения этой теоремы.
    \end{remark}
    \begin{theorem}
        Универсальный язык не разрешим (снова). Но теперь мы докажем это при помощи теоремы о рекурсии.
    \end{theorem}
    \begin{proof}
        Пусть разрешим, то есть есть разрешимая функция $U(p;x)$, которая возвращает 1, если $p(x)=1$ и 0 иначе. Напишем две штуки:
        \begin{verbatim}
            p(x):
                if u(p, x):
                    return 0;
                else:
                    return 1;
        \end{verbatim}
        Отличная программа. Теперь то же самое, но математически. Пусть $V(p;x)=\neg U(p;x)$. По теореме о рекурсии существует $p$ такое что $v(p;x)\equiv p(x)$. Если $p(x)=1$, то $V(p;x)=1$, значит $U(p;x)=0$, значит $p(x)\neq1$.
    \end{proof}
    \begin{theorem}
        $HALT_\varepsilon$ неразрешим.
    \end{theorem}
    \begin{proof}
        Пусть разрешим программой $h$. Тогда вот то, что приведёт нас к противоречию:
        \begin{verbatim}
            p(x):
                if x == epsilon && h(p):
                    while true {}
                else:
                    return 1
        \end{verbatim}
    \end{proof}
    \begin{theorem}
        Любое нетривиальное свойство перечислимых языков неразрешимо.
    \end{theorem}
    \begin{proof}
        Пусть есть свойство $A$ нетривиальное и разрешимое. Пусть у нас есть языки $Q$ и $R$ с полуразрешителями $q$ и $r$, где $Q\in A$ и $R\in A$. И пусть у нас есть $inA$~--- разрешитель $A$. Напишем такую программу:
        \begin{verbatim}
            p(x):
                if inA(p):
                    return r(x)
                else:
                    return q(x)
        \end{verbatim}
    \end{proof}
    \begin{remark}
        Перейдём уже к чему-то, что без теоремы о рекурсии плохо доказывается.
    \end{remark}
    \begin{theorem}[Теорема о неподвижной точке]
        Для любой всюду определённой вычислимой функции $f$ существует программа $p$, что $f(p)$ и $p$ ведут себя одинаково на любом вводе.
    \end{theorem}
    \begin{proof}
        Рассмотрим $f$. Напишем следующую программу:
        \begin{verbatim}
            p(x):
                q = f(p)
                return q(x)
        \end{verbatim}
    \end{proof}
    \begin{remark}
        Интересный факт: можно теорему о неподвижной точке рассматривать как базовую, а теорему о рекурсии~--- как её следствие.
    \end{remark}
    \begin{theorem}[Первая теорема Гёделя о неполноте]
        В любой достаточно богатой формально непротиворечивой системе существует верное недоказуемое утверждение.
    \end{theorem}
    \begin{remark}
        С нашей точки зрения аксиомы~--- это строки, и мы можем получать из одних строк другие (при помощи естественного вывода). И система непротиворечива, если нельзя одновременно доказать $\alpha$ и $\neg\alpha$. Причём тут нам пофиг вообще, какие у нас аксиомы нам единственное, что хочется~--- разрешимость. Т.е. по схеме аксиом и утверждению определить, подходит ли оно и по трём утверждениям выяснить, корректный ли modus ponens. В условиях вышеперечисленного мы можем проверить, корректное ли у нас доказательства.\\
        Что такое <<достаточно богатая>>~--- вопрос. В математической логике вводят конкретные аксиомы, причём достаточно малое их количество, но мы не жадные и скажем, что система достаточно богата, если в ней сформулировать утверждение <<программа на данном входе даёт единицу>>.
    \end{remark}
    \begin{proof}
        Рассмотрим такую программу:
        \begin{verbatim}
            p(x):
                s = "p(x) зависает"
                for y --- доказательство:
                    if y доказывает s:
                        return 1
        \end{verbatim}
        Если $p$ завершает работу, то она нашла доказательство, что она зависает. А у нас система непротиворечива. Значит она не завершает работу, но доказать мы это не можем.
    \end{proof}
    \begin{theorem}[Вторая теорема Гёделя о неполноте]
        Никакая достаточно богатая непротиворечивая система не может доказать свою непротиворечивость.
    \end{theorem}
    \begin{proof}
        Рассмотрим первую теорему. Рассмотрим программу, если $p(x)$ не зависает, то проблема, значит она зависает. А вот давайте попытаемся сформулировать эту же фразу в нашей системе. И если бы система могла доказать свою непротиворечивость, то их этого бы следовало, что $p$ зависает, а значит у этого было бы доказательство. Но его нет.
    \end{proof}
    \begin{remark}
        Давайте закончим с математической логикой и перейдём к алгоритмам сжатия.
    \end{remark}
    \begin{definition}
        $K(s)=\min\{\operatorname{len}(p)\mid p(\varepsilon)=s\}$ называется \textbf{колмогоровской сложностью}.
    \end{definition}
    \begin{theorem}
        Колмогоровская сложность невычислима.
    \end{theorem}
    \begin{proof}
        От противного. Рассмотрим программу
        \begin{verbatim}
            p(x):
                for s in Sigma^*:
                    if K(s) > len(p):
                        print(s)
                        return
        \end{verbatim}
        Мы найдём такую строку, что $K(s)$ больше длины $p$. То есть наименьшая длина программы, которая выводит $s$ больше длины $p$. Но $p$ же выводит $s$, противоречие. Причём такая строка $s$ точно существует, потому что \Verb|len(p)|~--- какое-то число, количество программ размера меньше либо равных его конечно, а значит и количество строк, у которых сложность меньше \Verb|len(p)| конечно.
    \end{proof}
    \begin{theorem}
        Пусть $f$ всюду определённая вычислимая функция такой что $\forall s~K(s)\geqslant f(s)$. Тогда $\exists c~f(s)\leqslant c$.
    \end{theorem}
    \begin{proof}
        Точно такое же, только вместо $K$ надо подставить $f$ в код. Поскольку $K(s)\geqslant f(s)>\operatorname{len}p$, ситуация та же самая. Тогда
        $$
        c=\max\{\operatorname{len}p,f(s):f(s)>\operatorname{len}p\}
        $$
        Если бы это множество было бесконечно, мы бы получили противоречие.
    \end{proof}
    \begin{definition}
        \textbf{Busy Beaver} (\textbf{усердный бобёр})~--- следующая функция.
        $$
        \operatorname{BB}(n)=\max\{\operatorname{len}s\mid\exists p:\operatorname{len}p\leqslant n,p(\varepsilon)=s\}
        $$
    \end{definition}
    \begin{theorem}
        Для любой всюду определённой вычислимой функции $f$ существует конечное число таких $n$, что $\operatorname{BB}(n)\leqslant f(n)$
    \end{theorem}
    \begin{remark}
        Доказательство остаётся читателю как домашнее задание.
    \end{remark}
    \paragraph{Машины Тьюринга.}
    \begin{remark}
        Что и зачем? Ну, смотрите. У нас есть математические модели и компьютер. Мы знаем, что они связаны тезисом Тьюринга~--- Чёрча, но не знаем, как и почему. И сегодня мы немного поговорим об этом. У нас будет машина Тьюринга, на которой в целом можно писать (хотя и не очень приятно), но получить из неё компьютер сложно. Но есть многодорожечная и многоленточная машины Тьюринга, которые более сложны, но уже можно понять, как они связаны с компьютерами. Но можно пойти от машины Тьюринга в другую сторону~--- в сторону более простых вычислителей: стековой и счётчиковой машин.
    \end{remark}
    \begin{definition}
        \textbf{Машина Тьюринга}~--- это объект из следующих сущностей: входной алфавит $\Sigma$, ленточный алфавит $\Pi\supset\Sigma$, есть символ $B\in\Pi\setminus\Sigma$. Есть множество состояний $Q$, стартовое состояние $s\in Q$, допускающее состояние $y\in Q$ и недопускающее $n\in Q$. А ещё есть функция перехода $\delta\colon Q\times\Pi\to Q\times\Pi\times\{\leftarrow,\rightarrow,\downarrow\}$.
    \end{definition}
    \begin{definition}
        \textbf{Состоянием машины Тьюринга} является пара из места головки на ленте, того, что на ленте в данный момент записано, и состояние $\in Q$. Обозначается это так:
        $$
        \alpha\#_q\beta
        $$
        $\alpha$~--- то что на ленте до головки, $\beta$~--- то что после (включая символ, где головка находится), $q$~--- состояние.
    \end{definition}
    \begin{remark}
        Начальное состояние выглядит как $\#_sX$, где $X\in\Sigma^*$.
    \end{remark}
    \begin{definition}
        Говорят, что машина Тьюринга \textbf{переходит из состояния} $\alpha c\#_qd\beta$ \textbf{в состояние} $STATE$, если
        \begin{itemize}
            \item $\delta(q;d)=(r;f;\rightarrow)$ и $STATE=\alpha cf\#_r\beta$.
            \item $\delta(q;d)=(r;f;\leftarrow)$ и $STATE=\alpha\#_rcf\beta$.
            \item $\delta(q;d)=(r;f;\downarrow)$ и $STATE=\alpha c\#_rf\beta$.
        \end{itemize}
        Обозначение $\vdash$. $\vdash^*$~--- \textbf{переходит из состояния в состояние за 0 или более шагов}.
    \end{definition}
    \begin{remark}
        Тут ещё случаи краёв надо рассмотреть, но мне лень.
    \end{remark}
    \begin{definition}
        \textbf{Язык машины Тьюринга} $m$~--- такое множество слов $L(m)$, что
        $$
        x\in L(m)\Leftrightarrow \#_sx\vdash^*\alpha\#_y\beta
        $$
    \end{definition}
    \begin{definition}
        $A$ \textbf{полуразрешим на М.Т.} если существует такая машина Тьюринга, что $L(m)=A$.
    \end{definition}
    \begin{definition}
        $A$ \textbf{разрешим на М.Т.} если существует такая машина Тьюринга, что $L(m)=A$ и для любого $x\in\Sigma^*$ $\#_sx\vdash^*\alpha\#_y\beta$ или $\#_sx\vdash^*\alpha\#_n\beta$
    \end{definition}
    \begin{claim}[Тезис Тьюринга~--- Чёрча]
        Языков разрешим на М.Т. тогда и только тогда, когда он разрешим.\\
        Языков полуразрешим на М.Т. тогда и только тогда, когда он полуразрешим.
    \end{claim}
    \begin{definition}
        \textbf{Многодорожечная машина Тьюринга}~--- то же самое, что и обычная М.Т., только
        $$
        \delta\colon Q\times\Pi^k\to Q\times\Pi^k\times\{\leftarrow,\rightarrow,\downarrow\}
        $$
        $k\geqslant 1$~--- количество дорожек.
    \end{definition}
    \begin{claim}
        Обычные машины Тьюринга эквивалентны многодорожечным.
    \end{claim}
    \begin{proof}
        В одну сторону очевидно, в другую~--- введём новый алфавит $\Pi'=\Pi^k$, $\Sigma'=\Sigma\times\{B\}^{k-1}$.
    \end{proof}
    \begin{definition}
        \textbf{Многоленточная машина Тьюринга}~--- то же самое что и многодорожечная, только смещение по каждой дорожке может отличаться:
        $$
        \delta\colon Q\times\Pi^k\to Q\times\Pi^k\times\{\leftarrow,\rightarrow,\downarrow\}^k
        $$
    \end{definition}
    \begin{claim}
        Многодорожечные машины Тьюринга эквивалентны многоленточным.
    \end{claim}
    \begin{proof}
        В одну сторону очевидно, в другую~--- сделаем $2k$ дорожек, на нечётных будет то, что на лентах многоленточной. На чётных будет только один символ $*$~--- он будет показывать, где находится головка на многоленточной машине. Как симулировать на таком устройстве многоленточную машину? В текущем состоянии будут помнится, на каких лентах уже встретила головку и какой символ она видела под головкой. Это увеличит количество состояний где-то в $(|\Pi|+1)^k$ раз (хотя на самом деле ещё больше), и наша машина будет идти направо до тех пор, пока не встретит все головки, тем самым запоминая их все, после чего делая переходы.\\
        Заметим, что при такой симуляции $t$ шагов на многоленточной машине превращаются в $O(t^2)$, что много, но всё ещё полином.
    \end{proof}
    \begin{remark}
        Теперь как, собственно, на такой штуке симулировать компьютер? Ну, легко. У нас есть регистры, стек, оперативка и программы. Мы читаем инструкцию, декодируем (т.к. инструкций конечное число), если там арифметика~--- сделаем, если память~--- ну, заведём себе лишнюю ленту, на которой будем считать разные вещи, и вот возьмём и просто дойдём до позиции $i$ в памяти (за $i$ шагов), и сделаем с памятью нужное действие.
    \end{remark}
    \begin{claim}
        Машина Тьюринга, которая имеет бесконечную только в одну сторону ленту и не пишет символ $B$ эквивалентна обычной машине.
    \end{claim}
    \begin{proof}
        От написания $B$ избавится легко: заведём себе $B'$, будем писать только его и трактовать так, как обычная машина трактовала $B$.\\
        Теперь избавимся от двусторонне-бесконечной ленты. Возьмём нашу ленту, и поломаем посередине пополам. В начало каждой половины запишем особый символ. Тепер добавляем в множество состояний то, на какой мы половине, а в переходы~--- взаимодействие с новым символом, который перекидывает нас с одной половины на другую. А также для состояний во второй половине инвертируем направление движения.
    \end{proof}
    \begin{claim}
        Автомат с одним стеком равносилен контекстно-свободной грамматике.
    \end{claim}
    \begin{claim}
        Автомат с двумя стеками равносилен машине Тьюринга.
    \end{claim}
    \begin{proof}
        Кладём одно слово в один стек, другое~--- переворачиваем и кладём в другой. Дальше понятно.
    \end{proof}
    \begin{definition}
        \textbf{Счётчиковая машина}~--- $\Sigma$, множество состояний (стартовое и множество терминальных) и функция переходов
        $$
        \delta\colon Q\times(\Sigma\cup\varepsilon)\times\mathbb B^k\to Q\times\{+1,-1,0\}^k
        $$
        Ещё нельзя уменьшать нулевые счётчики.
    \end{definition}
    \begin{remark}
        То есть у нас есть $k$ счётчиков, мы сравниваем их с нулём и в результате увеличиваем на один, уменьшаем на 1 или не меняем.
    \end{remark}
    \begin{claim}
        Два стека равносильны трём счётчикам.
    \end{claim}
    \begin{proof}
        У нас есть стек, там что-то лежит. Возьмём $b\geqslant |\Pi|$. Будем считать, что содержимое стека~--- число в системе счисления с отнованием $b$. Тогда нам надо уметь убирать цифру и добавлять. То есть мы должны делить с остатком на $b$, умножать на $b$ и прибавлять число от 0 до $|\Pi|$.\\
        Учимся делить~--- делаем цикл из $b$ состояний, которые устроены по кругу и каждое вычитаем 1, если это легально. Только одно прибавляет 1 у другому (изначально нулевому) числу. Тогда там будет частное. Остаток~--- то самое состояние, когда мы закончили. Умножение со сложением~--- аналогично.
    \end{proof}
    \begin{claim}
        $k\geqslant2$ равносильны двум счётчикам.
    \end{claim}
    \begin{proof}
        Возьмём простые числа. Все счётчики конвертируем в один так:
        $$
        2^{cnt_1}3^{cnt_2}\cdots p_k^{cnt_k}
        $$
        Как проверять заданный счётчик на равенство нулю? Поделить с остатком на соответствующее простое мы уже умеем~--- благо у нас есть второй счётчик, чтобы делить.
    \end{proof}
    \begin{remark}
        Отлично, а в чём вообще прикол машины Тьюринга? А прикол в том, что машина Тьюринга очень локально меняет строку~--- она меняет только начало одной и конец второй. Очень удобно рассматривать асимптотику.
    \end{remark}
    \begin{definition}
        \textbf{Машина Маркова}~--- вычислитель из одного строкового регистра, в котором изначально исходная строка. И есть у вычислителя \underline{упорядоченные} инструкции одного из трёх типов:
        \begin{itemize}
            \item $\alpha\to\beta$.
            \item $\alpha\to YES$.
            \item $\alpha\to NO$.
        \end{itemize}
        Выполняется следующее: на каждом шаге идём по списку правил по порядку. Если левая часть встречается как подстрока, заменяем первое вхождение на правую часть правила и запускаемся заново. Если же первое подошедшее правило~--- это $\alpha\to YES$ или $\alpha\to NO$, то происходит допуск или недопуск, а строка не меняется. Если нам важно не допуск/недопуск, можно заменить правила $\alpha\to YES$ и $\alpha\to NO$ на $\alpha\to STOP$
    \end{definition}
    \begin{claim}
        Машины Маркова равносильны машинам Тьюринга.
    \end{claim}
    \begin{proof}
        В одну сторону~--- тезис Тьюринга~--- Чёрча. Теперь проэмулируем машину Тьюринга на машине Маркова. Возьмём наше описание состояния машины Тьюринга $\alpha\#_s\beta$ и будем пытаться описывать его. Для начала вставим на последнюю позицию правил $\varepsilon\to\#_s$. Оно будет применено один раз в начале, и всё. Теперь для каждого перехода в машине Тьюринга вставим правило по типу $c\#_qd\to cf\#_r$ или какие там правила есть. После этого поставим правила с краёв, а ещё вставим $\#_y\to YES$ и $\#_n\to NO$.
    \end{proof}
    \begin{remark}
        Мы подходили о всем языкам через порождение и распознавание. Распознавание мы научились, а как порождать?
    \end{remark}
    \begin{definition}
        \textbf{Формальная грамматика нулевого класса} состоит из
        $\Sigma$~--- алфавит терминалов, $N$~--- нетерминалы, $S\in N$~--- стартовый нетерминал, $P\subset N^+\times(\Sigma\cup N)^*$~--- правила.
    \end{definition}
    \begin{definition}
        Говорят, что $\xi$ \textbf{порождает} $\eta$ \textbf{за один шаг} ($\xi\Rightarrow\eta$), если $\xi$ можно представить в виде $\xi_1\alpha\xi_2$, а $\eta$~--- в виде $\xi_1\beta\xi_2$ и есть правило $\alpha\to\beta\in P$.
    \end{definition}
    \begin{definition}
        Язык грамматики нулевого класса~--- $L(\Gamma)=\{w\in\Sigma^*\mid S\Rightarrow^*w\}$
    \end{definition}
    \begin{claim}
        Язык перечислим является тогда и толко тогда, когда существует грамматика нулевого класса, которая его распознаёт.
    \end{claim}
    \begin{proof}
        Языки грамматик можно перечислить~--- будем в порядке возрастания длины перебирать все возможные выводы. Если вывод кончается словом, напечатаем его.\\
        Почему наоборот? Возьмём тезис Тьюринга~--- Чёрча, возьмём машину Тьюринга, и попытаемся сделать грамматику, которая порождает тот де язык. Давайте наше порождение будет состоять из трёх фаз. Сначала мы породим нетерминалы $\char`\^$ и $\$$:
        $$
        S\to\char`\^Z\$
        $$
        Потом из $Z$ мы породим все слова, в которых где-то есть $\#_y$
        \begin{align*}
            &Z\to cZ\mid c\in\Pi\\
            &Z\to Zc\mid c\in\Pi\\
            &Z\to \#_y
        \end{align*}
        А теперь мы будем в обратном порядке мы будем симулировать машину Тьюринга. Пусть у нас было правило $\delta(q,c)=(r;d;\leftarrow)$, то есть мы могли преобразовать $a\#_qc$ в $\#_rad$. А мы делаем в обратном порядке, мы добавляем $\#_rad\to a\#_qc$. Похожим образом происходит всё, если у нас не $\leftarrow$, а что-то другое.\\
        И теперь последнее.
        \begin{align*}
            &\char`\^\#_s\to\Delta\\
            &\Delta c\to\boxed{c}\Delta\mid c\in\Sigma
        \end{align*}
        При этом $\boxed c$~--- терминал. Если мы боимся, что могли печататься пустые символы, то добавим ещё
        \begin{align*}
            &\Delta\to\star\\
            &\star B\to\star\\
            &\star\$\to\varepsilon
        \end{align*}
    \end{proof}
    \begin{definition}
        \textbf{Универсальный язык для грамматик нулевого класса}
        $$
        U_{\Gamma0}=\{(\Gamma;w)\mid S\Rightarrow^*w\}
        $$
    \end{definition}
    \begin{claim}
        Универсальный язык для грамматик нулевого класса перечислим, но не разрешим.
    \end{claim}
    \paragraph{Неразрешимые задачи, не связанные с языками.}
    \begin{example}
        Проблема соответствия Поста.\\
        Дано число $n$ и строки $\alpha_1,\ldots,\alpha_n,\beta_1,\ldots,\beta_n$. Требуется найти $k>0$ и последовательность индексов $i_1,\ldots,i_k$, чтобы
        $$
        \alpha_{i_1}\cdots\alpha_{i_k}=\beta_{i_1}\cdots\beta_{i_k}
        $$
    \end{example}
    \begin{definition}
        \textbf{Язык проблемы соответствия Поста} ($PCP$)~--- множество входных данных, для которых решение существует.
    \end{definition}
    \begin{theorem}
        $PCP$ не разрешим.
    \end{theorem}
    \begin{proof}
        Чтобы это доказать, сведём $U_{TM}\leqslant_mPCP$. $U_{TM}$~--- универсальный язык Тьюринг-машин. Простоты ради введём $PCP_1$ и сведём так:
        $$
        U_{TM}\leqslant_mPCP_1\leqslant_mPCP
        $$
    \end{proof}
    \begin{definition}
        $PCP_1$~--- то же самое, что $PCP$, но в решении $i_1=1$.
    \end{definition}
    \begin{lemma}
        $$
        PCP_1\leqslant_m PCP
        $$
    \end{lemma}
    \begin{proof}
        Давайте добавим символ $\#$ в алфавит и сконструируем новые $\alpha$ и $\beta$ так: во-первых, мы добавим $\alpha_0=\alpha_1$ и $\beta_0=\beta_1$. После этого в каждую $\alpha$ мы будем вписывать решётку после каждого символа, а в $\beta$~--- перед. За одним исключением: $\alpha_0$ будет иметь решётку перед каждым символом и в своём конце тоже решётку будет иметь. Тогда начать мы сможем только с $\alpha_0$. А продолжать мы можем чем угодно. Надо только решётку с конца убрать, так что добавим $\alpha_{n+1}=\$$, $\beta_{n+1}=\#\$$. Указанное преобразование, очевидно, вычислимо и, очевидно, показывает требуемое сведение.
    \end{proof}
    \begin{theorem}
        $U_{TM}\leqslant_mPCP_1$.
    \end{theorem}
    \begin{proof}
        У нас на входе машина Тьюринга и слово. Хочется построить ПСП, такую что у неё будет решение тогда и только тогда, когда указанная машина допускает указанное слово.\\
        Ну, смотрите. У нас есть некоторые конфигурации, которые в Машине Тьюринга переходят друг в друга. Так вот что мы сделаем: добавим $\vdash$ в алфавит и будем считать, что мы из $\alpha$ и $\beta$ пытаемся собрать вывод. Только $\alpha$ будет опережать $\beta$ на одно состояние\\
        С чего начать? Ну,
        $$\alpha_1=\char`\^\#_sx\vdash\qquad\qquad\beta_1=\char`\^ $$
        Что мы умеем? Мы умеем приписывать одинаковый символ: $\alpha_c=c,\beta_c=c$. Можем точно также приписывать $\vdash$. Что мы ещё умеем? Мы умеем следующее:
        $$\alpha_i=\#_rad\qquad\qquad \beta_i=a\#_qc$$
        Это в том случае, если $a\#_qc\vdash a\#_rc$. Аналогично с $\leftarrow$. Ещё у нас может быть правило с $B$, тогда вместо одного символов у нас будет $\vdash$. Всё это продолжается, пока не достигнем допускающего состояния. А дальше надо как-то равнять $\alpha$ и $\beta$. Ну, вот так можно
        $$
        \alpha_i=\#_y\qquad\qquad\beta_i=c\#_y
        $$
        $$
        \alpha_i=\#_y\qquad\qquad c\beta_i=\#_y
        $$
        $$
        \alpha_i=\$\qquad\qquad \beta_i=\#_y\vdash\$
        $$
        Понятно, что тут существенно, что у нас именно $PCP_1$.\\
        Если решение есть, мы его найдём. Если решения нет, мы от решётки не избавимся никак.
    \end{proof}
    \subsection{Свойства контекстно-свободных грамматик.}
    \begin{remark}
        Тут внезапно окажется, что очень многие хорошие свойства КС-грамматик не разрешимы. Но не все.
    \end{remark}
    \begin{example}
        Пусть
        $$A=\{(\Gamma_1;\Gamma_2)\mid L(\Gamma_1)\cap L(\Gamma_2)\neq\varnothing\}$$
        Сведём $A\leqslant_mPCP$. Пусть у нас есть $PCP$. Возьмём
        $$
        \Sigma'=\Sigma\cup\{\boxed1;\boxed2;\ldots;\boxed n\}
        $$
        Если хочется, чтобы не зависело от ввода, закодируем эти символы как бинарные строки из $\boxed0$ и $\boxed1$.\\
        Введём следующие КС-грамматики:
        \begin{align*}
            S&\to\alpha_1 A\boxed1\\
            S&\to\alpha_2 A\boxed2\\
            &\vdots\\
            S&\to\alpha_n A\boxed n\\
            A&\to\alpha_1 A\boxed1\\
            A&\to\alpha_2 A\boxed2\\
            &\vdots\\
            A&\to\alpha_n A\boxed n\\
            A&\to\varepsilon\\
        \end{align*}
        А вторая грамматика~--- аналогично. В первой грамматике можно получить $\alpha_{i_1}\cdots\alpha_{i_k}\boxed{i_k}\cdots\boxed{i_1}$, во второй~--- аналогично. Если можно получить одинаковое слово, значит решается и $PCP$. И наоборот.
    \end{example}
    \begin{definition}
        Если есть список слов $L=\{\alpha_1;\ldots;\alpha_n\}$, то язык, задаваемый КС:
        \begin{align*}
            S&\to\alpha_1 A\boxed1\\
            S&\to\alpha_2 A\boxed2\\
            &\vdots\\
            S&\to\alpha_n A\boxed n\\
            A&\to\alpha_1 A\boxed1\\
            A&\to\alpha_2 A\boxed2\\
            &\vdots\\
            A&\to\alpha_n A\boxed n\\
            A&\to\varepsilon\\
        \end{align*}
        называется \textbf{языком списка} $L$.
    \end{definition}
    \begin{property}
        Дополнение к языку списка контекстно-свободно.\\
        Остаётся как ДЗ читателю.
    \end{property}
    \begin{definition}
        \textbf{Полимино}~--- связный (по сторонам) набор единичных квадратов на плоскости.
    \end{definition}
    \begin{definition}
        \textbf{Язык TILING\textsubscript{4}}~--- язык замощения четверти плоскости заданными типами полимино.\\
        То есть дают какой-то набор полимино, и требуется замостить четверть плоскости ими (повороты и отражения запрещены).
    \end{definition}
    \begin{property}
        $\overline{TILING_4}$ не разрешим (а следовательно $TILING_4$ не разрешим).
    \end{property}
    \begin{proof}
        Сведём $U_{TM}\leqslant_m\overline{TILING_4}$.\\
        Зафиксируем достаточно больше $c$ (потом выберем, чему оно равно). Разобьём нашу четверть плоскости по горизонтали на полоски из $c$ строк, потом 3 строк, потом снова $c$, снова 3 и так далее. А по вертикал разобьём просто на полоски размера $c$. Сейчас мы сделаем такой набор, что если машина Тьюринга допускает слово, то замостить нельзя, а иначе можно.\\
        Пусть $c\times c$ будет задавать кусочек конфигурации машины Тьюринга, а ещё у нас будут переходники следующих размеров $3\times c$, $3\times 2c$ и $3\times 3c$. В нижней полоске напишем $\#_sx_1x_2\cdots x_nBBB\cdots$. Отсюда мы умеем только делать что-то вида
        $$
        d\#_qx_2\cdots x_nBBB
        $$
        То есть выберем $c$ больше, чем размер алфавита плюс количество состояний (то есть граница квадрата $c\times c$ кодирует состояние или символ). Введём следующие полимино:
        \begin{enumerate}
            \item Тип $\#_s$: имеет снизу и слева гладкие стороны, сверху~--- $\#_s$, справа~--- 1.
            \item Тип $x_1$: снизу гладкая, слева совпадает с $\#1$, справа~--- 2, сверху $x_1$.
            \item Тип $x_2$: снизу гладкая, слева совпадает с $x_1$, справа~--- 3, сверху $x_2$.
            \item $\vdots$
            \item Тип $B$: снизу гладкая, слева совпадает с $x_n$, справа~--- $n+1$, сверху $B$.
        \end{enumerate}
        Этими домино единственным образом можно замостить низ.\\
        Дальше для каждого символа $d$ будет полимино, у которого снизу выступ под $d$, сверху $d$, в справа~--- рисунок, который для всех символов одинаковый, но отличается от нижних полимино. И ещё для каждого символа будет то же самое, но с выступом слева, чтобы можно было втыкать символы и друг рядом с другом, и слева.\\
        {\bf\LARGE Кто при помощи TikZ или Asymptote сделает сюда рисунок, тот будет очень крут.\\}
        Осталось закодировать переходы. Они понятно как кодируются: у нас есть переходник, у которого снизу символ и сверху тот же символ, а есть переходник под каждое правило машины Тьюринга. Разве что надо ещё сделать так, чтобы нельзя было по вертикали соединять без переходника, но это легко решается.\\
        Осталось лишь приделать к $\#_n$ возможность повторить весь ряд, а к $\#_y$~--- ничего не приделать.
    \end{proof}
\end{document}